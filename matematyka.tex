\documentclass[a4paper,12pt]{article}
\usepackage[MeX]{polski}
\usepackage[utf8]{inputenc}
\usepackage{algorithmic}

%opening
\title{\TeX}
\author{Paweł Kumorowski}

\begin{document}
\maketitle

\begin{abstract}
Ćwiczenia z La\TeX a
\end{abstract}

\section{Polecenie do wykonania}
Zapisz w TeX'u poniższe wzory matematyczne, zwróć uwagę na numerowanie tych wzorów oraz odwołania do nich.

\begin{equation}
\lim\limits_{n \to \infty} \sum_{ik=1}^n \frac{1}{k^{2}} = \frac{\pi^{2}}{6}
\end{equation}

\begin{equation}
\prod_{i=2}^{n=i^2}=\frac{\lim\nolimits^{n \to 4}(1+\frac{1}{n})^n}{\sum k(\frac{1}{n})}
\end{equation}

Łatwo równanie 1 jest doprowadzić do 2

\begin{equation}
\int_2^\infty\frac{1}{\log_2x}dx=\frac{1}{x}\sin x=1-\cos^2(x)
\end{equation}

\begin{equation}
\left[\begin{array}{cccc}
a_{11} & a_{12} & \ldots & a_{1K}\\
a_{21} & a_{22} & \ldots & a_{2K}\\
\vdots & \vdots & \ddots &\vdots \\
a_{K1} & a_{K2} & \ldots & a_{KK}\\
\end{array}\right]
*
\left[\begin{array}{c}
x_1 \\
x_2 \\
\vdots \\
x_K
\end{array} \right]
=
\left[\begin{array}{c}
b_1 \\
b_2 \\
\vdots \\
b_K
\end{array}\right]
\end{equation}

\begin{equation}
(a_1=a_1(x))\wedge(a_2=a_2(x))\wedge\dots\wedge(a_k=a_k(x))\Rightarrow(d=d(u))
\end{equation}

\begin{equation}
[x]_A=\{y\in U:a(x)=a(y),\forall a \in A \} \textnormal{, where the central object} \  x \in U
\end{equation}

\begin{equation}
g(u,r)=\{v \in U: \frac{card\{IND(u,v)\}}{card\{A\}|} \geq r \}
\end{equation}

\begin{equation}
\textnormal{where,} \ IND(u,v)= \{ a \in A:a(u)=a(v) \}
\end{equation}

\begin{equation}
T:[0,1] \times [0,1] \rightarrow [0,1]
\end{equation}

\begin{equation}
cos(2\theta )=cos^2\theta - sin^2\theta
\end{equation}

\begin{equation}
\lim\limits_{x \to \infty} \textnormal{exp}(-x)=0
\end{equation}

\begin{equation}
\frac{n!}{k!(n-k)!}={n\choose k}
\end{equation}

\begin{equation}
P\Bigg(A=2\Bigg|\frac{A^2}{B} > 4 \Bigg)
\end{equation}

\begin{equation}
S^{c_i}(a)=\frac{(\bar{C}^a_i-\hat{C}^a_i)^2}{Z_{\bar{C}{^a_i}^2}+Z_{\hat{C}{^a_i}^2}},a \in A
\end{equation}

\begin{equation}
C^a_i=\{a(u):u\in U \ and \ d(u) =c_i\}
\end{equation}

\begin{equation}
A_{c_i}(a)=C^a_i \wedge_\varepsilon \{U\backslash C^a_i\}
\end{equation}

\begin{equation}
w(u_q,v_p)w(u_q,v_p)+\frac{|a(u_q)-a(v_p)|}{(max\_attr_a-min\_attr_a)\ast \varepsilon}
\end{equation}

\begin{equation}
c'_{ij}= \left\{ 
\begin{array}{c}
c_{ij} \ \textnormal{gdy} \ d(x_i) \neq d(x_j) \\
\phi \textnormal{gdy} d(x_i)=d(x_j).
\end{array} \right.
\end{equation}

\begin{algorithmic}
\STATE{Procedure} 
\STATE{Input data}
\STATE{$A'\leftarrow \emptyset$}
\STATE{$iter \leftarrow 0$}
\FOR{i=1,2,...,card\{A\}}
\FOR{j=1,2...,k}
\STATE{$F^{c_j}(a)=F^{c_j}_i(a)$}
\IF{$a\notin A'$}{
\item{$A'\leftarrow a$}
\item{$iter\leftarrow iter + 1$}
}
\IF{$iter=fixed\ number\ of\ the\ best\ genes$}{
\item{BREAK}
}
\ENDIF
\ENDIF
\ENDFOR
\IF{$iter=fixed\ number\ of\ the\ best\ genes$}{
\item{BREAK}
}
\ENDIF
\ENDFOR
\RETURN{$A'$}
\end{algorithmic}

$$
\left. \begin{array}{c}
S^{c_1}_1(a)\>S^{c_1}_2(a)>...>S^{c_1}_{card\{A\}}(a) \\
S^{c_2}_1(a)\>S^{c_2}_2(a)>...>S^{c_2}_{card\{A\}}(a) \\
\vdots \\
S^{c_k}_1(a)\>S^{c_k}_2(a)>...>S^{c_k}_{card\{A\}}(a) \\
\end{array}\right.
$$

\end{document}