\documentclass[a4paper,12pt]{article}
\usepackage[MeX]{polski}
\usepackage[utf8]{inputenc}

%opening
\title{\TeX}
\author{Paweł Kumorowski}

\begin{document}
\maketitle

\begin{abstract}
Ćwiczenia z La\TeX a
\end{abstract}

\section{Polecenie do wykonania}
Zapisz w TeX'u poniższe wzory matematyczne, zwróć uwagę na numerowanie tych wzorów oraz odwołania do nich.

\begin{equation}
\lim\limits_{n \to \infty} \sum_{ik=1}^n \frac{1}{k^{2}} = \frac{\pi^{2}}{6}
\label{eq:rownanie}
\end{equation}

\begin{equation}
\prod_{i=2}^{n=i^2}=\frac{\lim\nolimits^{n \to 4}(1+\frac{1}{n})^n}{\sum k(\frac{1}{n})}
\label{eq:rownanie}
\end{equation}

Łatwo równanie 1 jest doprowadzić do 2

\begin{equation}
\int_2^\infty\frac{1}{\log_2x}dx=\frac{1}{x}\sin x=1-\cos^2(x)
\label{eq:rownanie}
\end{equation}

\begin{equation}
\left[\begin{array}{cccc}
a_{11} & a_{12} & \ldots & a_{1K}\\
a_{21} & a_{22} & \ldots & a_{2K}\\
\vdots & \vdots & \ddots &\vdots \\
a_{K1} & a_{K2} & \ldots & a_{KK}\\
\end{array}\right]
*
\left[\begin{array}{c}
x_1 \\
x_2 \\
\vdots \\
x_K
\end{array} \right]
=
\left[\begin{array}{c}
b_1 \\
b_2 \\
\vdots \\
b_K
\end{array}\right]
\label{eq:rownanie}
\end{equation}

\begin{equation}
(a_1=a_1(x))\wedge(a_2=a_2(x))\wedge\dots\wedge(a_k=a_k(x))\Rightarrow(d=d(u))
\label{eq:rownanie}
\end{equation}

\begin{equation}
[x]_A=\{y\in U:a(x)=a(y),
\label{eq:rownanie}
\end{equation}

\end{document}