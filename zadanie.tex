\documentclass[a4paper,12pt]{article}
\usepackage[MeX]{polski}
\usepackage[utf8]{inputenc}

%opening
\title{Wydział Matematyki i Informatyki Uniwersytetu Warmińsko-Mazurskiego}

\author{Paweł Kumorowski}

\begin{document}
\maketitle
Wydział Matematyki i Informatyki UWM
\begin{abstract}

\end{abstract}
Wydział Matematyki i Informatyki Uniwersytetu Warmińsko-Mazurskiego (WMiI) --- Wydział Uniwersytetu Warmińsko-Mazurskiego w Olsztynie oferujący studia na dwóch kierunkach:
\begin{itemize}
\item Informatyka
\item Matematyka
\end{itemize}
w trybie studiów stacjonarnych i niestacjonarnych. Ponadto oferuje studia podyplomowe.
\newline
Wydział zatrudnia 8 profesorów, 14 doktorów habilitowanych, 53 doktorów i 28 magistrów.
\tableofcontents
\begin{itemize}
\item{1 Misja}
\item{2 Opis kierunków}
\item{3 Struktura organizacyjna}
\item{4 Władze wydziału}
\item{5 Historia wydziału}
\item{6 Nowa siedziba wydziału}
\item{7 Adres}
\item{8 Przypisy}
\item{9 Linki zewnętrzne}
\end{itemize}

\section{Misja}
Misją wydziału jest:
\begin{itemize}
\item{Kształcenie matematyków zdolnych do udziału w rozwijaniu matematyki i jej stosowania w innych
działach wiedzy i w praktyce;}
\item{Kształcenie nauczycieli matematyki, nauczycieli matematyki z fizyką a także nauczycieli informatyki;}
\item{Kształcenie profesjonalnych informatyków dla potrzeb gospodarki, administracji, szkolnictwa oraz życia społecznego;}
\item{Nauczanie matematyki i jej działów specjalnych jak statystyka matematyczna, ekonometria,
biomatematyka, ekologia matematyczna, metody numeryczne; fizyki a w razie potrzeby i podstaw
informatyki na wszystkich wydziałach UWM.}
\end{itemize}

\section{Opis kierunków}
Na kierunku Informatyka prowadzone są studia stacjonarne i niestacjonarne:
\begin{itemize}
\item{studia pierwszego stopnia – inżynierskie (7 sem.), sp. inżynieria systemów informatycznych, informatyka ogólna}
\item{studia drugiego stopnia – magisterskie (4 sem.), sp. techniki multimedialne, projektowanie systemów informatycznych i sieci komputerowych}
\end{itemize}
Na kierunku Matematyka prowadzone są studia stacjonarne:
\begin{itemize}
\item{studia pierwszego stopnia – licencjackie (6 sem.), sp. nauczanie matematyki, matematyka stosowana}
\item{studia drugiego stopnia – magisterskie (4 sem.), sp. nauczanie matematyki, matematyka stosowana}
\end{itemize}
oraz studia niestacjonarne:
\begin{itemize}
\item{studia drugiego stopnia – magisterskie (4 sem.), sp. nauczanie matematyki}
\end{itemize}
Państwowa Komisja Akredytacyjna w dniu 19 marca 2009r. oceniła pozytywnie jakość kształcenia na kierunku Matematyka, natomiast w dniu 12 marca 2015r. oceniła pozytywnie jakość kształcenia na kierunku Informatyka.

\section{Struktura organizacyjna}
Katedry:
\begin{itemize}
\item{Katedra Algebry i Geometrii}
\item{Katedra Analizy i Równań Różniczkowych}
\item{Katedra Analizy Zespolonej}
\item{Katedra Fizyki i Metod Komputerowych}
\item{Katedra Fizyki Relatywistycznej}
\item{Katedra Informatyki i Badań Operacyjnych}
\item{Katedra Matematyki Dyskretnej i Teoretycznych Podstaw Informatyki}
\item{Katedra Matematyki Stosowanej}
\item{Katedra Metod Matematycznych Informatyki}
\item{Katedra Multimediów i Grafiki Komputerowej}
\end{itemize}
Ośrodki:
\begin{itemize}
\item{Ośrodek Informatyczno-Sieciowy}
\end{itemize}

\section{Władze Wydziału}
Dziekan i prodziekani na kadencję 2016-2020:
\begin{itemize}
\item{Dziekan: dr hab. Jan Jakóbowski, prof. UWM}
\item{Prodziekan ds. nauki: prof. dr hab. Aleksy Tralle, prof. zw.}
\item{Prodziekan ds. studenckich: dr Aleksandra Kiślak-Malinowska}
\item{Prodziekan ds. kształcenia: dr Piotr Artiemjew}
\end{itemize}

\section{Historia Wydziału}
Wydział Matematyki i Informatyki został utworzony 1 września 2001 roku, po powołaniu dwa lata wcześniej
Uniwersytetu Warmińsko-Mazurskiego, ale jego korzenie sięgają lat 50. XX w. Decyzję o powołaniu Wydziału
podjął Senat UWM w dniu 10 lipca 2001 r. Badania związane z zastosowaniami matematyki rozpoczęły się
wraz z powołaniem w 1950 roku Zakładu Matematyki w Zespołowej Katedrze Fizyki, a od 1951 roku –
Katedry Statystyki Matematycznej ówczesnej Wyższej Szkoły Rolniczej, przemianowanej w 1972 r. na
Akademię Rolniczo Techniczną. Natomiast kształcenie matematyczne i badania w dziedzinie matematyki
zapoczątkowane zostały wraz z utworzeniem w roku 1969 Wyższej Szkoły Nauczycielskiej (od 1974 r. pod nazwą Wyższa Szkoła Pedagogiczna). Wydział jest kontynuatorem działań Katedry Zastosowań Matematyki ART oraz Instytutu Matematyki i Fizyki WSP.
Od 27 kwietnia 2009 Wydziałowi przyznano prawo do nadawania stopnia naukowego doktora w dziedzinie
nauk matematycznych w dyscyplinie matematyka.

\end{document}